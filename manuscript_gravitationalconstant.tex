\documentclass[prb,preprint]{revtex4-1} 
% The line above defines the type of LaTeX document.
% Note that AJP uses the same style as Phys. Rev. B (prb).

% The % character begins a comment, which continues to the end of the line.

\usepackage{amsmath}  % needed for \tfrac, \bmatrix, etc.
\usepackage{amsfonts} % needed for bold Greek, Fraktur, and blackboard bold
\usepackage{graphicx} % needed for figures
\usepackage{epstopdf}


\begin{document}

% Be sure to use the \title, \author, \affiliation, and \abstract macros
% to format your title page.  Don't use lower-level macros to  manually
% adjust the fonts and centering.

\title{A scenario for a Planckian standard kilogram}
% In a long title you can use \\ to force a line break at a certain location.

\author{Michael Ebersold}
\email{michael.ebersold@uzh.ch} 
% If there were a second author at the same address, we would put another 
% \author{} statement here.  Don't combine multiple authors in a single
% \author statement.
\author{Prasenjit Saha}
\email{psaha@physik.uzh.ch}
% Please provide a full mailing address here.
\affiliation{Physik-Institut, University of Zurich, Winterthurerstrasse 190, 8057 Zurich, Switzerland}

% See the REVTeX documentation for more examples of author and affiliation lists.

\date{\today}

\begin{abstract}
  
\end{abstract}

\maketitle


\begin{thebibliography}{99}
\bibitem{BIPM16} International Bureau of Weights and Measures, ``DRAFT 9th edition of the SI Brochure", BIPM, (2016)

\bibitem{Chao15} L. S. Chao, S. Schlamminger, D. B. Newell, J. R. Pratt, F. Seifert, X. Zhang, G. Sineriz, M. Liu and D. Haddad, ``A LEGO Watt balance: An apparatus to determine a mass based on the new SI", \textit{American Journal of Physics}, 83:11:913--922 (2015).

\bibitem{Tomilin1999} D. K. Tomilin, ``Natural Systems of Units. To the Centenary Anniversary of the Planck System,'' 
Proceedings of the XXII Workshop on high Energy Physics and field theory, 290 (1999).  

\bibitem{Planck99} M. Planck, ``\"{U}ber irreversible {S}trahlungsvorg\"{a}nge,'' 
Sitzungsberichte der K\"{o}niglich Preussischen Akademie der Wissenschaften zu Berlin, 479--480 (1899).  

\bibitem{BIPM} Bureau International des Poids et Mesures, Calibration and measurement services: Mass, \url{<https://www.bipm.org/utils/common/pdf/cms_m.pdf>}, (2015).

\bibitem{CODATA16} P. J. Mohr, D. B. Newell and B. N. Taylor, ``CODATA recommended values of the fundamental physical constants: 2014*, \textit{Reviews of Modern Physics}, 88(3):035009 (2016).

\bibitem{Quinn14} T. Quinn, C. Speake, H. Parks and R. Davis, ``The BIPM measurements of the Newtonian constant of gravitation $G$'', \textit{Philosophical Transactions of the Royal Society of London A: Mathematical, Physical and Engineering Sciences}, 372(2026) (2014).

\bibitem{Newman14} R. Newman, M. Bantel, E. Berg and W. Cross, ``A measurement of $G$ with a cryogenic torsion pendulum'', \textit{Philosophical Transactions of the Royal Society of London A: Mathematical, Physical and Engineering Sciences}, 372(2026) (2014).

\bibitem{Luo09} J. Luo, W. Liu, L. Tu, C. Shao, L. Liu, S. Yang, Q. Li and Y. Zhang, ``Determination of the Newtonian gravitational constant $G$ with time-of-swing method'', \textit{Phys. Rev. Lett.}, 102:240801 (2009).

\bibitem{Schlamminger06} S. Schlamminger, E. Holzschuh, W. Kündig, F. Nolting, R. E. Pixley, J. Schnurr and U. Straumann, ``Measurement of Newton's gravitational constant", \textit{Phys. Rev. D}, 74:082001 (2009).

\bibitem{Parks10} H. V. Parks and J. E. Faller, ``Simple pendulum determination of the gravitational constant", \textit{Phys. Rev. Lett.}, 105:110801 (2010).

\bibitem{Rosi14} G. Rosi, F. Sorrentino, L. Cacciapuoti, M. Prevedelli and G. M. Tino, ``Precision measurement of the Newtonian gravitational constant using cold atoms", \textit{Nature}, 510:518--521 (2014).

\bibitem{Ch-Dalsgaard05} J. Christensen-Dalsgaard, M.~P. Di Mauro,
  H. Schlattl and A. Weiss, ``On helioseismic tests of basic
  physics'', \textit{MNRAS}, 356:587--595 (2005).

\bibitem{Loughridge13} R. Loughridge and D. Y. Abramovitch, ``A tutorial on laser interferometry for precision measurements,'' 
American Control Conference, 3686--3703 (2013). 

\bibitem{esopycoffee} Fast and easy-to-implement Markov Chain Monte Carlo with the emcee package, ESO PyCoffee (2014),
\url{<http://eso-python.github.io/ESOPythonTutorials/ESOPythonDemoDay8_MCMC_with_emcee.html>}.

\bibitem{vanRavenzwaaij16} D. van Ravenzwaji, P. Cassey and S. D. Brown, ``A simple introduction to Markov Chain Monte-Carlo sampling,'' 
Psychonomic Bulletin {\&} Review, 1--12 (2016). 

\bibitem{corner} D. Foreman-Mackey, ``corner.py: Scatterplot matrices in Python,'' 
The Journal of Open Source Software (10), (2016).  


\end{thebibliography}

\end{document}

@INPROCEEDINGS{2016APS..APR.R5001S,
   author = {{Schlamminger}, S.},
    title = ``{Measurements of the gravitational constant - why we need new ideas}'',
booktitle = {APS April Meeting Abstracts},
     year = 2016,
    month = mar,
   adsurl = {http://adsabs.harvard.edu/abs/2016APS..APR.R5001S},
  adsnote = {Provided by the SAO/NASA Astrophysics Data System}
}

@ARTICLE{2005MNRAS.356..587C,
   author = {{Christensen-Dalsgaard}, J. and {Di Mauro}, M.~P. and {Schlattl}, H. and 
     {Weiss}, A.},
    title = ``{On helioseismic tests of basic physics}'',
  journal = {\mnras},
 keywords = {Sun: fundamental parameters, Sun: helioseismology, Sun: interior, cosmological parameters},
     year = 2005,
    month = jan,
   volume = 356,
    pages = {587-595},
      doi = {10.1111/j.1365-2966.2004.08477.x},
   adsurl = {http://adsabs.harvard.edu/abs/2005MNRAS.356..587C},
  adsnote = {Provided by the SAO/NASA Astrophysics Data System}
}
