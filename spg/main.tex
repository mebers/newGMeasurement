\documentclass[aps,prb,12pt]{revtex4-1} 

\def\unit#1{\;{\rm#1}}

\begin{document}

\title{Planck and the Extra-terrestrials\\
       or what the new SI means for astronomy}

\author{Prasenjit Saha} \email{psaha@physik.uzh.ch}
\affiliation{Physik-Institut, University of Zurich\\
  Winterthurerstr.~190, 8057~Zurich, Switzerland}

\begin{abstract}
New freedom.
\end{abstract}

\maketitle

In 1899 Planck \cite{Planck1899}.
Let $m=\sqrt{hc/G}$.
$f \rightarrow G$, $b\rightarrow h$, $a\rightarrow h/k$
Comes to $55\rm\,\mu g$.

Not the first \cite{Tomilin1999} but most prescient.  SI
\cite{Jeckelmann_2018}.  Leaves out $G$.  Also candela, sievert.

Astros use Planckian units occasionally (with $\hbar$) but generally
not.  Stories of cross-sections in gallons/Mpc.\footnote{A gallon/Mpc
  is about a kilogram, but the exact value depends on which gallon.}
Why?

Solar mass in kg would be bad.  SI in astro has been a non-starter.
But spacecraft dynamics uses $GM$ in SI.

New SI new freedoms.  eV not a new unit.  Density as
$\!\unit{eV}\unit{m}^{-3}$ is fine.  Anarchy?  No, because conversions
only through physical constants.

For astros, don't need to express solar mass in kg.  But we'll get to
that.

First Light-second: au and pc.  Again, not a new unit.  Round like Pfund.

In the astronomy literature one finds many named units (\AA ngström,
gauss, jansky) that are simply powers of 10 times SI units, which need
not detain us here.  There are also parody units such as
$\!\unit{gallon}\unit{Mpc}^{-1}$ for cross-sections (it depends on the
type of gallon, but is about a kilobarn), which would need an article
to themselves.  The only remaining unit that needs discussing is the
optical magnitude scale.  Optical magnitudes are a measure of
brightness going back to classical times, but like other units they
have been progressively redefined, and are now considered
SI-traceable.  A zero-magnitude source (such as the star Vega)
corresponds to a flux of roughly $10^{10}$ photons
$\!\unit{m}^{-2}\unit{s}$ in a typical spectral band.  More precisely,
zero magnitude is $5.480 \times10^{10}$ photons
$\!\unit{m}^{-2}\unit{s}$ per logarithmic spectral interval, with each
5~mag being 100 times fainter.  Since modern optical cameras are
invariably counters of photons, it makes sense to simply use photon
fluxes rather than optical magnitudes, with ``rounded'' magnitudes
used for historical comprehension.

Gravity-second $GM/c^3$ (pulsars).  Strange formulations.
Dimensionless velocity ok.  But density $\rho$ as $s^{-2}$ seems
strange.  But $(G\rho)^{-1/2}$ is a time scale.  And gravity-second
per second is a power.  Basically, avoid explicit $G$ in pure
gravitational phenomena.  Explicit $G$ only when gravity and non-grav
processes both appear.

In the 120 years since Planck speculated about it, we have yet to make
contact with non-human or extra-terrestrial cultures, and ask them if
they indeed share Planck's views on units.  What {\em has} happened,
however, is that physical constants have turned out to have meanings
not even Planck in 1899 had imagined.  As we all know, the following
year Planck himself discovered that $h$ (formerly known as $b$) was
about quantization.  A few years later the formal relation between
mass and energy through $c^2$ turned out to be real.  Starting in the
1920s $(\hbox{Planck mass})^{-2}\times(\hbox{proton mass})^3$ was
discovered to be the mass scale of stars.  Later in the 20th century,
$2e/h$ and $h/e^2$ turned out to be macroscopically observable as
Josephson's constant and von~Klitzing's constant.  And just in the
last few years, $c^5/G$ has been observed to be the power-output scale
for merging black holes.  We can only wonder what more surprises await
us.

\def\aj{AJ}
\def\apjl{ApJL}
\def\aap{A\&A}
\def\mnras{MNRAS}
\def\pasj{PASJ}
\def\jgr{JGR}

\bibliographystyle{apsrev4-1}
\bibliography{refs.bib}

\end{document}




