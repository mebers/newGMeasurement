\documentclass[aps,prb,12pt]{revtex4-1} 

\def\unit#1{\;{\rm#1}}

\begin{document}

\title{Planck and the Extra-terrestrials\\
       or what the new SI means for astronomy}

\author{Prasenjit Saha} \email{psaha@physik.uzh.ch}
\affiliation{Physik-Institut, University of Zurich,
  Winterthurerstr.~190, 8057~Zurich, Switzerland}

\begin{abstract}
New freedom.
\end{abstract}

\maketitle

In 1899, during his struggle to understand blackbody radiation but
about a year before he found the key, Planck proposed a curious set of
units.\cite{Planck1899} In these natural units, as he called them, the
unit of mass $m=\sqrt{hc/G}$ which comes to about $55\rm\,\mu g$.  The
length unit is $h/(mc)$, the time unit is $h/(mc^2)$, while the
temperature unit is $mc^2/k$.  (Reading the original today, one needs
to substitute $f \rightarrow G$, $b\rightarrow h$, $a\rightarrow
h/k$.)  Planck then declared that these units would be recognized by
all cultures as fundamental, even extraterrestrial or non-human ones
(ausserirdische und aussermenschliche Culturen).

Planck was by no means the only person to try 

Not the first \cite{Tomilin1998} but most prescient.  SI
\cite{Jeckelmann_2018}.  Leaves out $G$.  Also candela, sievert.

Astros use Planckian units occasionally\cite{magicenv} (with $\hbar$)
but generally not.\cite{dodd2011}  Why?

Solar mass in kg would be bad.  SI in astro has been a non-starter.
But spacecraft dynamics uses $GM$ in SI.

New SI new freedoms.  eV not a new unit.  Density as
$\!\unit{eV}\unit{m}^{-3}$ is fine.  Anarchy?  No, because conversions
only through physical constants.

For astronomy, the good news is that the SI no longer requires you to
express the solar mass in kg --- but we'll get to that.  We remark
first that distances in light-seconds are compliant with the new SI.
Like the electronvolt, a light-second is not itself a unit, it is an
abbreviated way of referring to the equivalent light travel time.  On
the other hand, which astronomer would wish to lose the cultural
legacies of the astronomical unit of length (au), and the parsec?
Fortunately, metrology and cultural legacies can be reconciled.  To
see how, we only to see how modern German has kept the old unit of a
Pfund, but has rounded its meaning to exactly $500\unit{g}$.  The au
and parsec have serendipitously round values: $1\unit{au}\simeq
c\times500\unit{s}$ to better than 1\%, while a parsec is just 3\%
over $c\times10^8\unit{s}$.  

In the astronomy literature one finds many named units (\AA ngström,
gauss, jansky) that are simply powers of 10 times SI units, which need
not detain us here.  There are also parody units such as
$\!\unit{gallon}\unit{Mpc}^{-1}$ for cross-sections (it depends on the
type of gallon, but is about a kilobarn), which would need an article
to themselves.  The only remaining unit that needs discussing is the
optical magnitude scale.  Optical magnitudes are a measure of
brightness going back to classical times, but like other units they
have been progressively redefined, and are now considered
SI-traceable.  A zero-magnitude source (such as the star Vega)
corresponds to a flux of roughly $10^{10}$ photons
$\!\unit{m}^{-2}\unit{s}$ in a typical spectral band.  More precisely,
zero magnitude is $5.480 \times10^{10}$ photons
$\!\unit{m}^{-2}\unit{s}$ per logarithmic spectral interval, with each
5~mag being 100 times fainter.  Since modern optical cameras are
invariably counters of photons, it makes sense to simply use photon
fluxes rather than optical magnitudes, with ``rounded'' magnitudes
used for historical comprehension.

Now we come to the kilogram, or avoiding the kilogram when necessary.
Spacecraft dynamics has long used the so-called solar mass parameter
$GM_\odot$ in $\!\unit{m}^3\unit{s}^{-2}$, which is known to eight
significant digits in Newtonian gravity, two more with general
relativity.  In pulsar timing, $GM_\odot/c^3$ in seconds is usual.  In
the new SI, by analogy with electronvolts, we can meaningfully measure
mass as $GM/c^3$ in ``gravity-seconds''.  The solar mass is about
5~micro-(gravity)-seconds.  This does not mean that kilograms are to
be avoided in astrophysics.  Rather, gravity-seconds can be useful
where kilograms would needlessly propagate the uncertainty in $G$,
which is typically the case for orbital processes.  Together with
distances in light-seconds and dimensionless velocities,
gravity-seconds allow for some elegant simplifications in the
description of all kinds of orbital processes, classical and
relativistic.\cite{2020PASP..132b1001S} Some constructions arise, that
seem {\em very} strange at first: in particular density in
gravity-seconds per cubic light-seconds, or power in gravity-seconds
per second.  But these are not really unphysical at all.  In
gravitational phenomena, a density $\rho$ is always associated with a
time scale $(G\rho)^{-1/2}$, so density as frequency squared does make
sense.  As for dimensionless power in gravity-seconds per second, this
is just power in units of the well-known Planck power $c^5/G$, which
is the luminosity scale in gravitational waves for merging black
holes, irrespective of their mass.

In the 120 years since Planck speculated about it, we have yet to make
contact with non-human or extra-terrestrial cultures, and ask them if
they indeed share Planck's views on units.  What {\em has} happened,
however, is that physical constants have turned out to have meanings
not even Planck in 1899 had imagined.  As we all know, the following
year Planck himself discovered that $h$ (formerly known as $b$) was
about quantization.  A few years later the formal relation between
mass and energy through $c^2$ turned out to be real.  Starting in the
1920s $(\hbox{Planck mass})^{-2}\times(\hbox{proton mass})^3$ was
discovered to be the mass scale of stars.  Later in the 20th century,
$2e/h$ and $h/e^2$ turned out to be macroscopically observable as
Josephson's constant and von~Klitzing's constant.  And just in the
last few years, $c^5/G$ has been observed to be the power-output scale
for merging black holes.  We can only wonder what more surprises await
us.

\def\aj{AJ}
\def\apjl{ApJL}
\def\aap{A\&A}
\def\mnras{MNRAS}
\def\pasp{PASP}

\bibliographystyle{apsrev4-1}
\bibliography{refs.bib}

\end{document}




