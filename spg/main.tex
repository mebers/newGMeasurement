\documentclass[aps,prb,12pt]{revtex4-1} 

\def\unit#1{\;{\rm#1}}

\begin{document}

\title{Planck and the Extra-terrestrials\\
       or what the new SI means for astronomy}

\author{Prasenjit Saha} \email{psaha@physik.uzh.ch}
\affiliation{Physik-Institut, University of Zurich,
  Winterthurerstr.~190, 8057~Zurich, Switzerland}

%\begin{abstract}
%\end{abstract}

\maketitle

In 1899, during his struggle to understand blackbody radiation but
about a year before he found the key, Planck proposed a curious set of
units.\cite{Planck1899} In these natural units, as he called them, the
unit of mass $m$ is $\sqrt{hc/G}$ which comes to about $55\rm\,\mu g$.
The length unit is $h/(mc)$, the time unit is $h/(mc^2)$, while the
temperature unit is $mc^2/k$.  (Reading the original today, one needs
to substitute $f \rightarrow G$, $b\rightarrow h$, $a\rightarrow
h/k$.)  Planck then declared that these units would be recognized by
all cultures as fundamental, even extraterrestrial or non-human ones
(ausserirdische und aussermenschliche Culturen).

Planck was not the only one to propose a system of units based on
fundamental constants,\cite{Tomilin1998} but he was the most
prescient.  In Planck's system there is no prototype mass, length,
time, or temperature.  Instead there are dimensional coefficients that
appear in multiple equations of physics, which define the units
implicitly.  In the 21st century the International System of units
concluded that implicit definition of units is not a bug, it's a
feature.  In 2018 the SI decided that only the second shall be defined
by a particular physical process (a spectral line in Cs).  All other
units are defined implicitly, through defined values of physical
constants, including (but of course) Planck's constant.  Any
applicable equation of physics can be used for the realization of
units.\cite{Jeckelmann_2018}

The reforms in the SI were not especially designed to communicate with
extraterrestrials.  The decision to adopt implicit definitions was
pragmatic, in that explicit definitions were sometimes overtaken by
technology.  In particular, the kilogram is best realized as a derived
electrical unit, with mass measured by way of the quantum Hall effect
and the Josephson effect.  That would be sternly frowned upon by the
old SI, but the new SI is cheerfully agnostic about it.  The new SI
also removes conflicts with some common scientific usages.  For
example, the electronvolt is not an SI unit.  However, if we
understand mass in electronvolts as a shorthand for the value of mass
times $c^2/e$ in volts, there is no conflict with the new SI.

Planckian units can be useful useful for theoretical work, provided
one changes to standard units for comparing with observable
quantities.\cite{magicenv} Meanwhile, the community of scientists
engaged with extraterrestrial things have steadfastly resisted
adopting SI units.  Even the proponents of SI units among astronomers
\cite{dodd2011} have not advocated a complete change to the SI.  Why
is that?  The answer may lie to some extent in sociological factors,
but there has been a scientific reason.  The key is the constant that
Planck included but the new SI does not: $G$.  The gravitational
constant is uniquely weak and uniquely hard to measure, and even now
is known to only four significant digits.\cite{2018Natur.560..562S}
The so-called solar mass parameter $GM_\odot$ is known to eight
significant digits in Newtonian gravity, two more with general
relativity.  But if you split that product into $G$ and $M_\odot$, in
order to write the mass of the Sun in kilograms, your value of the
gravitational field will be only good to four significant digits.  You
cannot navigate a spacecraft on four significant digits.  End of story
for SI in astronomy.  Or so it used to be.

After the reforms, however, the SI no longer requires you to express
the solar mass in kg --- but we'll get to that.  We remark first that
distances in light-seconds are compliant with the new SI.  Like the
electronvolt, a light-second can be understood as not itself a unit,
but an abbreviated way of referring to the equivalent light travel
time.  On the other hand, which astronomer would wish to lose the
cultural legacies of the astronomical unit of length (au), and the
parsec?  Fortunately, metrology and cultural legacies can be
reconciled.  To see how, we need only to see how modern German has
kept the old unit of a Pfund, but has rounded its meaning to exactly
$500\unit{g}$.  The au and parsec have serendipitously round values:
$1\unit{au}\simeq c\times500\unit{s}$ to better than 1\%, while a
parsec is just 3\% over $c\times10^8\unit{s}$.  Many astronomical
applications do not require 1\% precision, and there ``rounded'' au
and parsecs would work just fine.

In the astronomy literature one finds many further units (\AA ngström,
gauss, jansky) that are simply powers of 10 times SI units, and need
not detain us here.  There are also parody units such as
$\!\!\unit{gallons}\unit{Mpc}^{-1}$ for cross-sections (it depends on
the type of gallon, but is about a kilobarn), which would need an
article to themselves.  The only remaining unit that needs discussing
is the optical magnitude scale.  Optical magnitudes are a measure of
brightness going back to classical times, but like other units they
have been progressively redefined, and are now considered
SI-traceable.  A zero-magnitude source (such as the star Vega)
corresponds to a flux of roughly $10^{10}$ photons
$\!\unit{m}^{-2}\unit{s}$ in a typical spectral band.  More precisely,
zero magnitude is $5.480 \times10^{10}$ photons
$\!\unit{m}^{-2}\unit{s}$ per logarithmic spectral interval, with each
5~mag being 100 times fainter.  Since modern optical cameras are
invariably counters of photons, it makes sense to simply use photon
fluxes rather than optical magnitudes, with ``rounded'' magnitudes
used for historical comprehension.

Now we come to the kilogram, or avoiding the kilogram when necessary.
Spacecraft dynamics has long used $GM_\odot$ in
$\!\unit{m}^3\unit{s}^{-2}$.  In pulsar timing, $GM_\odot/c^3$ in
seconds is usual.  In the new SI, by analogy with electronvolts, we
can meaningfully measure mass as $GM/c^3$ in ``gravity-seconds''.  The
solar mass is about 5~micro-(gravity)-seconds.  This does not mean
that kilograms are to be avoided in astrophysics.  Rather,
gravity-seconds can be useful where kilograms would needlessly
propagate the uncertainty in $G$, which is typically the case for
orbital processes.  Together with distances in light-seconds and
dimensionless velocities, gravity-seconds allow for some elegant
simplifications in the description of all kinds of orbital processes,
classical and relativistic.\cite{2020PASP..132b1001S} Some
constructions arise that seem {\em very} strange at first: in
particular density in gravity-seconds per cubic light-seconds, or
power in gravity-seconds per second.  But these are not really
unphysical at all.  In gravitational phenomena, a density $\rho$ is
always associated with a time scale $(G\rho)^{-1/2}$, so density as
frequency squared does make sense.  As for dimensionless power in
gravity-seconds per second, this is just power in units of the
well-known Planck power $c^5/G$, which is the luminosity scale in
gravitational waves for merging black holes, irrespective of their
mass.

In the 120 years since Planck speculated about it, we have yet to make
contact with non-human or extra-terrestrial cultures, and ask them if
they indeed share Planck's views on units.  What {\em has} happened,
however, is that physical constants have turned out to have meanings
not even Planck in 1899 had imagined.  As we all know, the following
year Planck himself discovered that $h$ (formerly known as $b$) was
about quantization.  A few years later the formal relation between
mass and energy through $c^2$ turned out to be real.  Starting in the
1920s $(\hbox{Planck mass})^{-2}\times(\hbox{proton mass})^3$ was
discovered to be the mass scale of stars.  Later in the 20th century,
$2e/h$ and $h/e^2$ turned out to be macroscopically observable as
Josephson's constant and von~Klitzing's constant.  And just in the
last few years, $c^5/G$ has been observed to be the power-output scale
for merging black holes.  We can only wonder what more surprises await
us.

\def\aj{AJ}
\def\apjl{ApJL}
\def\aap{A\&A}
\def\mnras{MNRAS}
\def\pasp{PASP}

\bibliographystyle{apsrev4-1}
\bibliography{refs.bib}

\end{document}




